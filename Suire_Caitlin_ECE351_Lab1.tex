%%%%%%%%%%%%%%%%%%%%%%%%%
%                       %
%  Caitlin Suire        %
%  ECE 351-53           %
%  Lab 1                %
%  1/26/21              %
%                       %
%%%%%%%%%%%%%%%%%%%%%%%%%


\documentclass{article}
\usepackage[utf8]{inputenc}

\title{Lab1}
\author{Caitlin Suire}
\date{January 2021}



\documentclass[11pt,a4]{article}
\usepackage[utf8]{inputenc}
\usepackage{fullpage}
\usepackage{hyperref}
\usepackage{listings}
\usepackage{xcolor}
\usepackage{graphicx}
\definecolor{codegreen}{rgb}{0,0.6,0}
\definecolor{codegray}{rgb}{0.5,0.5,0.5}
\definecolor{codeblue}{rgb}{0,0,0.95}
\definecolor{backcolour}{rgb}{0.95,0.95,0.92}
\lstdefinestyle{mystyle}{
backgroundcolor=\color{backcolour},
commentstyle=\color{codegreen},
keywordstyle=\color{codeblue},
numberstyle=\tiny\color{codegray},
stringstyle=\color{codegreen},
basicstyle=\ttfamily\footnotesize,
breakatwhitespace=false,
breaklines=true,
captionpos=b,
keepspaces=true,
numbers=left,
numbersep=5pt,
showspaces=false,
showstringspaces=false,
showtabs=false,
tabsize=2
}
\lstset{style=mystyle}
\title{Title}

\author{Caitlin Suire}
\date{\today}


\begin{document}

\maketitle

\section{Questions and Deliverables}

I am most excited for the 400 level design courses and power courses in my degree.\\ % New line without indention or spacing
The course I have enjoyed the most so far would have to be 210 and 212 and maybe 320 as well. These courses really gave me the gist of electric circuitry and its complexity. 
\\ % Finally learned how to space paragraphs here 

\noindent The expectations, instructions, and delverables all seem very clear. I like the structure of these documents and handouts compared to other labs. This makes things quite easy to understand and follow. I will need to read and learn more about LaTex and Python, but if the handouts continue like this, it should be straight forward for me to pick up on. Thanks so much. 

\end{document}
